\subsection{Client-Server Agents and the QR Code}
\label{ssec:clientServer}
QR code can be read quickly by many modern cell phones. It is used to take a piece of 
information from a transitory media and put it in to your cell phone. 
It may give you details about a URL, vCard, plain text, etc.

We propose an application able to read a QR code (currently most smartphones contain it).
The smartphone application will be able to connect with the \blockchaincarnetwork in order
to do different operations (See Table~\ref{table:operations} to know a list of operations). 

In particular, the smartphone application will be denoted as the client $\Client$. Such an application, 
actually, when is running connects with a miner, denoted as \Server, within the \blockchaincarnetwork.


%\subsubsection{Setting a QR Code in a Car}
%\label{sssec:settingQR}
The QR code must be generated by dealerships who manufactures vehicles and they must place the code
in both: physically in the vehicle and virtually in the invoice.

We have established that the QR code must be labeled in some part of the car and it must content,  
in json format, the following data: 
\textit{id=vehicle identification number}, 
\textit{trademark}, 
\textit{model}, 
\textit{class}, 
\textit{version}, 
\textit{number of cylinders}, and
among \textit{others}. These attributes are own of a vehicle, and they do not 
change with time. An example in JSON format is as follows:
\begin{table}[h]
    \centering
    \caption{Genesis information about the vehicle}
    \begin{tabular}{lll}
       \{&         			&    							\\
         & id:        		& "1FMYU02Z97KA580G2", 			\\
         & tradeMark: 		& "abcd", 						\\
         & model:     		& "2012", 						\\
         & class:     		& "auto", 						\\
         & version:   		& "TA XLS 4X2 I4 TELA 4 CIL", 	\\
         & cylinders: 		& "L4" 							\\
       \}& 		        	& 								\\
       ::& \textit{others}	&								\\
    \end{tabular}
    \label{table:genesisInfo}
\end{table}

The information in the QR code will be in plain-text because such an information is not 
sensitive and anyone can obtain such data of any car.

%\subsubsection{Client-Server Parts}
%\label{sssec:readingQR}



%Throughout this document, we refer to \QR code as the information obtained after a reading 
%process. 
