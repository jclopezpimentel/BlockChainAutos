\subsection{Fraud Description}
From 2011, paper bills stopped being legal proof of ownership for cars. An electronic bill had to be generated by a seller in order to legally report a car selling took place.
The process of the selling is described bellow:
\begin{itemize}
    \item The buyer checks that no legal problems are involved with the vehicle. An online check can be done with the plates numbers. 
    \item The buyer makes the car payment
    \item The seller generates an electronic bill for the value of the car and acknowledges payment from the buyer.
    \item The seller gives possession of the car to the buyer.
    \item The change of ownership is notified to the city government. 
    \item The old plates are discarded and a new set of plates is given the new owner.
\end{itemize}

This work flow has a vulnerability between steps $2$ and $3$. The electronic bill generated in step $2$ can be generated even without the existence of a physical car. A malicious seller could generate multiple bills and give them to multiple people before they realize the car doesn't exist physically. A certain level of \textit{trust} must exist between buyer and seller for the exchange to happen. And this \textit{trust} can be maligned. 

This fraud is specially common on car sales between private owners for a couple reasons. The first one is the high value of this commodity, which makes it a low risks - high stakes trade-off for fraudsters. The other one comes with the definition of car: A portable, efficient, and maneuverable transportation vehicle that can be easily removed from the transaction scene and hidden afterwards.

Due to the lack of immediate retaliation available to the criminal, he has a time frame where the fraud could be repeated multiple times, even if finally, the car exchange takes place with an Innocent buyer afterwards.