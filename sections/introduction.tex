\section{Introduction}
The paperless culture extends throughout the world. 
From the birth of the scanner to the invention of cloud computing, 
technology has been making its way into business processes 
with the aim of facilitating operations and reducing the use of paper, 
for both economic and environmental reasons. 

Companies and organizations who seek to become paperless can increase the automation of their processes and at the same time use a minimum amount of paper.
At a bigger scale, this is not a complete process and in some cases is not even encouraged.
Even today, 
in some communities where an electronic bill is required when selling goods, 
it has to be printed afterwards and then officially signed, 
via holograms or stamps, 
for it to become a legal proof. 

The combination of electronic and physical documents introduces some application problems, 
like a lack of clarity in the process of passing ownership.
Specially troublesome is the case when the goods are expensive, like automobiles.
Sometimes, a car is the most valuable possession a person has, 
making the selling process a subject of constant scrutiny 
because if a fraud is committed in the exchange, 
the affected party can lose a big fraction of its patrimony.

To solve our challenge, smart properties is a recent
technology that can be used to this. Smart property is a specialization of 
smart contracts whose support is the block chain technology.
%Examples could include physical property such as vehicles, phones or houses. 
A smart contract is a computer protocol intended to digitally facilitate,
verify, or enforce the negotiation or performance of a
contract, using transactions. These transactions are traceable
and irreversible.

%In this paper, 
%we will present a distributed model solution based on smart properties technology 
%to handle automobile related transactions 
%that offers a secure, useful and easy to use system on which car ownership can be
%handled, validated and updated.

In this paper, 
we present a distributed model solution based on smart properties technology 
to handle automobile related transactions on which vehicle ownership, 
among others, can be handled, validated and updated through a 
secure and useful system with traceable and irreversible characteristics. 
Our model is presented as a lot of security protocols, each of them is shown
in Alice and Bob notation.

The paper is organized as follows: 
First, in Section~\ref{sec:background}
we look at recent technologies that can handle this kind of system,
including blockchain, smart contracts and smart properties.
Then, in Section~\ref{sec:outline}
we propose a model to control vehicle transactions through blockchain network platform 
that at the same time will prevent fraudulent operations. 
Our model starts by car dealerships including a QR code in vehicles. 
The code could be read by users who can use it to do transactions through \blockchaincarnetwork. 
%Next, we present Alice and Bob notation to explain our general model; 
Sections~\ref{sec:client},~\ref{sec:getServices} and~\ref{sec:transactions} presents the model 
we are proposing with a step by step explanation. 
%the invoice supply-chain provenance. 
Finally, our conclusions and future work are given.