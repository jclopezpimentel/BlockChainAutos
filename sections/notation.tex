\subsection{Notation}
%As you can see, our stages imply the use of cryptography,  and security protocol techniques. 
% The following two sub-sections describe the notation used in the rest of
% the document. 

% \subsubsection{Host level notation}
% \label{sec:hostLevel}
Our notation has some similarities with Burrows et.al.~\cite{Burrows1990}. Let start with
cryptography, it has two main mechanisms:
\begin{itemize}
  \item Symmetric cryptography, the same key is used to encode and
    decode a message; by convention, symbol $\fat{m}_{K_{AB}}$ will be
    used to denote that a message $m$ has been cyphered under the key
    $K_{AB}$, which agents $A$ and $B$ know. \footnote{An agent is a 
        computer process that uses the client-server technology to 
        establish a network communication. Sometimes we will use the 
        terms agent or client indistinctly.} 
  \item Asymmetric cryptography, two keys are used to encode and
    decode messages; an agent (e.g. $A$) has a key pair
    (public $K_{pub(A)}$ and private $K_{priv(A)}$). One key used to
    encode and the other one to decode (reciprocally); usually, key
    $K_{priv(A)}$ is kept in secret. 
\end{itemize}
Table~\ref{table:conventions} summarizes previous notations:
\begin{table}[htb]
\footnotesize
\begin{center}
\caption{Conventions in types of messages and agents}
\label{table:conventions}
\begin{tabular}{|l|l|}
\hline
{\bf Abbreviation}& {\bf Description}                                   \\\hline\hline
$C$                 &  Agent, Client or Smartphone application able to \\
                    &  establish communication with other agent.         \\ 
$\Server$           &  A trusted agent, best known as trusted Miner-Server      \\
$K$                 &  Symmetric key                                      \\
$K_{AB}$            &  Symmetric key shared with agents $A$ and $B$       \\
$K_{pub(A)}$        &  Public key of Agent $A$                            \\
$K_{priv(A)}$       &  Private key of Agent $A$                           \\
$\fat{m}_{K}$       &  Message $m$ cyphered under key $K$                 \\
$m1\lnk m2$         &  $m1$ and $m2$ are concatenated using symbol $\lnk$ \\
$[m_1:: m_2]$       &  A list o messages, $m_1$ representing the first element\\
$N_a$, $N_b$, $N_c$ &  Nonces, which are random unguessable numbers      \\
                    &  never been used before.                           \\ \hline \hline
\end{tabular}
\end{center}
\end{table}
\normalsize

% \subsubsection{Notation at network level}
% \label{sec:networkLevel}
For easy representation issues we will use Alice and Bob notation to 
describe Client-Server communication, see table~\ref{table:NetConventions}.
%three views (see ): i) initial knowledge;
%ii) message sending and message reception; and iii) the local process
%representation.
\begin{table}[htb]
\footnotesize
\begin{center}
\caption{Conventions in network level}
\label{table:NetConventions}
\begin{tabular}{|l|l|}
\hline
{\bf Abbreviation}                      & {\bf Description}                    \\\hline\hline
\textbf{Initial knowledge:}             &                                      \\
$A : ik$                                &  Initially agent $A$ knows $ik$.    \\ \hline 
\textbf{Message Sending and Receiving:} &  At step $n$ agent $A$ sends         \\ 
$n. A \rightarrow B: m$                 &  message $m$ to agent $B$,\\
                                        &  which $B$ receives.      \\ \hline 
$n. A\rightarrow B: m$                  &  message $m2$ was generated \\ 
\hspace{5mm}\textbf{Local process:}     &  Between steps $n$ and $n2$,    \\ 
\hspace{5mm}$m2 = f(m)$                 &  by $B$ as a local process \\ 
$n2. B\rightarrow A: m2$                &  generated from $f(m)$.     \\ \hline 
\textbf{Broadcast:}                     &  $A$ broadcasts message to agents\\ 
$A \rightarrow [B, C, D]: m$            &  $B$, $C$ and $D$.\\ \hline \hline 
\end{tabular}
\end{center}
\end{table}
\normalsize


$$$$
