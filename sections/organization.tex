The paperless culture extends throughout the world. From the birth of the scanner to the invention 
of cloud computing, technology has been making its way into business processes with the aim of
facilitating operations and reducing the use of paper, for both economic and environmental reasons. 
To do that, companies seek to become paperless that can automate their processes with the 
minimum use of paper and with the highest automation.

In response to this, around the world there is an initiative for electronic invoice instead of the physical paper. 
Our interest is related to some problems that this initiative has caused in the 
purchase process of vehicles in Mexico. This is described as follows:

From 2011, paper bills stopped being legal proof of ownership for cars. An electronic bill had to be generated by a seller in order to legally report a car selling took place.
The process of the selling is described bellow:
\begin{enumerate}
    \item The buyer checks that no legal problems are involved with the vehicle. An online check can be done with the plates numbers. 
    \item The buyer makes the car payment
    \item The seller generates an electronic bill for the value of the car and acknowledges payment from the buyer.
    \item The seller gives possession of the car to the buyer.
    \item The change of ownership is notified to the city government. 
    \item The old plates are discarded and a new set of plates is given the new owner.
\end{enumerate}

This work flow has a vulnerability between steps $2$ and $3$. The electronic bill generated in step $2$ can be generated even without the existence of a physical car. A malicious seller could generate multiple bills and give them to multiple people before they realize the car doesn't exist physically. A certain level of \textit{trust} must exist between buyer and seller for the exchange to happen. And this \textit{trust} can be maligned. 

This fraud is specially common on car sales between private owners for a couple reasons. The first one is the high value of this commodity, which makes it a low risks - high stakes trade-off for fraudsters. The other one comes with the definition of car: A portable, efficient, and maneuverable transportation vehicle that can be easily removed from the transaction scene and hidden afterwards.

Due to the lack of immediate retaliation available to the criminal, he has a time frame where the fraud could be repeated multiple times, even if finally, the car exchange takes place with an Innocent buyer afterwards.

To solve our challenge, smart properties  is a recent technology that can be used to this. Smart property is those whose ownership is controlled via block chain technology using smart contracts~\cite{Tapscott2016}. Examples could include physical property such as vehicles, phones or houses. A smart contract is a computer protocol intended to digitally facilitate, verify, or enforce the negotiation or performance of a contract, using transactions. These transactions are trackable and irreversible. 

So, we propose a model to solve vehicle purchase fraud through blockchain network platform. In this paper we present our model consisting in including a QR code in vehicles by Dealerships and it can be read by users who can use it to do transactions through \blockchaincarnetwork. 
We have used Alice and Bob notation to explain our general model. We think that our proposal can be applied to other countries around the world who are looking for having a digital ecosystem and better
control over the transaction history that a vehicle can have throughout its life.

The rest of the paper is organized as follows: 
Section~\ref{sec:TheoreticalFramework} describes the notation used 
throughout the document and expounds the blockchain and smart contracts, 
which constitutes a fundamental part of our work; 
Section~\ref{sec:outline} presents the framework we are proposing. 
%the invoice supply-chain provenance. 
Section~\ref{sec:Services} explains step by step the proposal.
Finally, our conclusions are given.