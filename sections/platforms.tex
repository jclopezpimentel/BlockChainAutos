\subsection{Platforms}
The existence of smart contracts over a blockchain expand the use cases of this revolutionary technology above its original use in Bitcoin, \cite{nakamoto}. As alternative to Bitcoin, other crytocurrency systems are continuously in development, and in some cases, their network also have processing power, to be consumed as a service for the users\cite{uses}. Among the multiple platforms with processing power that allow smart contract interactions a completely autonomous system can be modeled, working as a Decentralized Autonomous System \cite{dao}.
Complete autonomy is not yet implementable 
due to the nonexistence of a sufficiently mature smart contract environment, \cite{daofail}. Other similar systems use this full autonomous environment, like Slock.it, Augur, etc \cite{sense}. 

Specially among them Ethereum \cite{ethereum} as a platform also provides a full system where users can run custom-scripted code inside their transactions, embracing the idea of smart contracts over a decentralized medium. But in contrast with the ones mentioned before, Ethereum gives full control over the degree of autonomy required as a back technology that supports the development of this environment can be utilized to built a smart contract management system \cite{lazy}. 

Other options, like Hawk \cite{hawk} also allows for a fully customizable system, adding more benefits, like transactional privacy.


