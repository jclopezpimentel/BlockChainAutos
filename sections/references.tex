% Can use something like this to put references on a page
% by themselves when using endfloat and the captionsoff option.
\ifCLASSOPTIONcaptionsoff
  \newpage
\fi



% trigger a \newpage just before the given reference
% number - used to balance the columns on the last page
% adjust value as needed - may need to be readjusted if
% the document is modified later
%\IEEEtriggeratref{8}
% The "triggered" command can be changed if desired:
%\IEEEtriggercmd{\enlargethispage{-5in}}

% references section

% can use a bibliography generated by BibTeX as a .bbl file
% BibTeX documentation can be easily obtained at:
% http://mirror.ctan.org/biblio/bibtex/contrib/doc/
% The IEEEtran BibTeX style support page is at:
% http://www.michaelshell.org/tex/ieeetran/bibtex/
\bibliographystyle{IEEEtran}
% argument is your BibTeX string definitions and bibliography database(s)
\bibliography{bib/paper}

\nocite{*}

% biography section
% 
% If you have an EPS/PDF photo (graphicx package needed) extra braces are
% needed around the contents of the optional argument to biography to prevent
% the LaTeX parser from getting confused when it sees the complicated
% \includegraphics command within an optional argument. (You could create
% your own custom macro containing the \includegraphics command to make things
% simpler here.)
%\begin{IEEEbiography}[{\includegraphics[width=1in,height=1.25in,clip,keepaspectratio]{mshell}}]{Michael Shell}
% or if you just want to reserve a space for a photo:

%\begin{IEEEbiography}{Juan-Carlos L\'opez-Pimentel}
%Biography text here.
%\end{IEEEbiography}


\begin{IEEEbiographynophoto}{Juan-Carlos L\'opez-Pimentel}
was born in Tuxtla Guti\'errez, Mexico. He holds a PhD and Master in Computer Science, both from the Tecnológico de Monterrey Campus Estado de Mexico. He has a Bachelor in Computer Systems Engineering from Institute of Technology in Tuxtla Gutierrez (in 2001). He was distinguished for being part of the Mexican National System of Researchers (2010-2013), Level C. His research interest is about Computer Security, Distributed Systems and Software Development. Currently he is a Research Professor at Universidad Panamericana, Campus Guadalajara.
\end{IEEEbiographynophoto}


% if you will not have a photo at all:
\begin{IEEEbiographynophoto}{Miguel Alcaraz}
Biography text here.
\end{IEEEbiographynophoto}
was born in Zapopan, Mexico. He received an MS degree in Optics in 2006 and his PhD in 2011 from Instituto Nacional de Astrofisica Optica y Electronica. At present, he works in the Engineering Department at Universidad Panamericana, Mexico. His research interests are focused on security and social networking.
% insert where needed to balance the two columns on the last page with
% biographies
%\newpage

\begin{IEEEbiographynophoto}{Leonardo J. Valdivia}
was born in Guadalajara, Mexico, in 1988. He received an MS degree in Telecommunications Engineering in 2014 and his PhD in 2017 from the University of Navarras School of Engineering (TECNUN).After spending 3 years working on software for the automotive sector and four years for railway sector, in 2017, he joined the Engineering Area at Universidad Panamericana. At present, his research interests are focused on safety and security interaction.
\end{IEEEbiographynophoto}


\begin{IEEEbiographynophoto}{Carolina del Valle Soto}
was born in Medellin, Colombia. She has a PhD. in Information Technology and Communications with a doctoral dissertation: Design, implementation and comparison of a new routing protocol for Wireless Sensor Networks. She studied a Master in Science of electronic engineering (Telecommunications) with a thesis named Development of a P2P network with DNS security. Carolina has a Bachelor in electronic engineering with a thesis in Design and construction of a photon counting systems. Carolina´s main skills in to apply and develop in the areas of telecommunications, specifically wired and wireless networks, programming and security. Carolina belongs to the National Research System, level C.
\end{IEEEbiographynophoto}


% You can push biographies down or up by placing
% a \vfill before or after them. The appropriate
% use of \vfill depends on what kind of text is
% on the last page and whether or not the columns
% are being equalized.

%\vfill

% Can be used to pull up biographies so that the bottom of the last one
% is flush with the other column.
%\enlargethispage{-5in}



% that's all folks
