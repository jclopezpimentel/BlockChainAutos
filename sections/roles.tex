\subsection{Roles}

According to the permissions the users will have five defined roles: 
\textbf{U}ser,
\textbf{D}ealership,
\textbf{O}wner,
\textbf{G}overnment, and
\textbf{H}elper.
Each user could hold one or more roles.
Table \ref{table:operations} pinpoints the type of operations that each role can do and the explanation
is as follows: 
\begin{itemize}
    \item Any \textbf{U}ser will have enough permissions to interact with the system in order to 
        obtain information on the vehicle (Section~\ref{sec:getServices}). This role is also able 
        to buy cars, but it must be authenticated. 
        These permissions will be granted to the user once he has established the secure channel. 
    \item The \textbf{D}ealership is able to manufacture vehicles because of that he is who creates the 
        genesis block.
    \item Every car will have one and only one \textbf{O}wner at a given time. The owner will be the only one 
        that has permission to trade ownership, but also to request to the government to change legal
        status. In addition, he can request and authorize to \textbf{H}elper to add some information 
        about a car.
    \item \textbf{G}overnment will have the ability to  \textit{change legal status} when the 
        vehicle is involved in legal situations. For that, it will be necessary to establish 
        different status: \textit{stolen}, \textit{arrested}, \textit{penalty}, \textit{registered owner}, \textit{plate}, 
        \textit{tax history}, etc.
    \item The \textbf{H}elper role is who receives a temporal permission from the \textbf{O}wner to set one 
        transaction on the block-chain. The time frame for the temporal permission will be authorized by 
        the \textbf{O}wner of the car. The car owner will set a special transaction to give the helper 
        the permission. The helper role will be used for transactions that change one or more data of the 
        car's properties. This will include official services, mechanical adjustments, big aesthetic 
        modifications and insurance payments.
\end{itemize}







\begin{table}[htb]
\footnotesize
    \begin{center}
    \caption{Type of operation}
    \label{table:operations}
        \begin{tabular}{|l|l|l|l|l|l|}
        \hline
        \textbf{Operation}          &\textbf{U}& \textbf{D}&\textbf{O}& \textbf{G}& \textbf{H}\\ \hline
        Create (genesis block)      &          & X         &          &           &           \\ \hline
        get information             & X        &           &          &           &           \\ \hline\hline
        sell                        &          &           & X        &           &           \\ \hline
        buy                         & X        &           &          &           &           \\ \hline\hline
        request change legal status &          &           & X        &           &           \\ \hline
        change legal status         &          &           &          & X         &           \\ \hline\hline
        request to add information  &          &           & X        &           &           \\ \hline
        add information             &          &           &          &           & X         \\ \hline
        authorize add information   &          &           & X        &           &           \\ \hline
        \end{tabular}
    \end{center}
\end{table}