\section{Outline}
\label{sec:outline}

%This section describes in words our general framework and describes
%the general notation used to explain the formalism.

In general there exists some characteristics we are looking for:
a) Provenance,
b) Transparency,
c) Traceability

\subsection{Outline for the Framework}
\label{subsec:proposal}
Figure~\ref{fig:flowChartFramework} illustrates a context diagram denoting
the general stages a client must follow to get or set information about a car. 
\begin{figure}[bt]
 %\begin{center}
  \centering
    \includegraphics[scale=0.7]{images/gralScheme.pdf}
        \caption{Diagram about the use of the framework from the client view}
    \label{fig:flowChartFramework}
 % \end{center}
\end{figure}

\begin{itemize}
  \item \textbf{Stage I} consists in getting information of a car through reading
    a QR code;
  \item \textbf{Stage II} consists in establishing a secure channel. We
    have used the TLS protocol because it is the standard in the
    e-commerce and it has been subject to a lot of verification proofs.  
  \item \textbf{Stage III} consists in to obtain services, authentication
    is not required.   
  \item \textbf{Stage IV:} consists in to obtain services or set transactions, 
    authentication is required.  
\end{itemize}

\subsection{Stage I: Reading a QR Code from a Car}
\label{sec:readingQR}

QR code can be read quickly by many modern cell phones. It is used to take a piece of information from a 
transitory media and put it in to your cell phone. It may give you details about a URL, vCard, plain text, etc.
We have established that the QR code is going labeled in some part of the car and it must content in json format 
the following data: 
\textit{license plate}, 
\textit{vehicle identification number}, 
\textit{trademark}, 
\textit{model}, 
\textit{class}, 
\textit{version}, and
\textit{number of cylinders}.
An example is as follows:

\begin{lstlisting}
    
  {
    plate:      "LVP6598", 
    id:         "1FMYU02Z97KA580G2", 
    tradeMark:  "Ford", 
    model:      "2012", 
    class:      "auto", 
    version:    "TA XLS 4X2 I4 TELA 4 CIL", 
    cylinders:  "L4"
  }
\end{lstlisting}

The information in the QR code will be in plain-text because such an information is not sensitive and anyone 
can obtain such data of any car.


\subsection{Stage II: Establishing a secure channel}
\label{sec:secureChannel}
Messages transmitted on the Internet are sensitive to be seen for
active or passive attackers who are able to manipulate them in order
to take advantage of the situation.

Transport Layer Security (TLS) and its predecessor, Secure Socket
Layer (SSL), is a \emph{cryptographic protocol} that works over the transport 
layer of TCP/IP. This protocol allows client/server agents to communicate 
across a hostile network like Internet; eventually, only the server is authenticated. 
Mutual authentication requires a public key infrastructure for clients. 

We have used the TLS protocol because, besides being the 
standard in e-commerce, the identity of the client is not
relevant while consulting information. However, for transactions that 
require authentication, then we have taken an additional step that we will 
explain later. Note that, the following   

Table~\ref{table:sslAndtlsProtocol} shows the protocol in Alice and Bob 
notation, it has been adapted from \cite{lopez13} and its explanation is as follows:
\begin{itemize}
  \item Initially, $\Server$ denotes a trusted miner server. $\Client$ denotes
    a user having read a QR code, and $\Servera$ denotes a server certification 
    authority. $\Server$ knows its public and private key; all (client and servers) 
    know their own identities. $\Servera$ knows all issued
    public keys. Here $\Key$ denotes a set of all public keys.
  \item Then, the following steps are carried out in order to
    establish the secure tunneling:
    \begin{itemize}
    \item \textbf{First step:} a client connects to a TLS server
      requesting a secure connection (in plain-text) and presents a list
      of supported CipherSuites (ciphers and hash functions),
      $[Lc]$. Each session is identified by a session id $N_{C}$. 
    \item \textbf{Second step:} from list, the server picks the
      strongest cipher and hash function that it also supports
      ($f([Lc])$) and notifies the client of the decision. The server
      sends back its identity in the form of a digital certificate and
      in plain-text. The certificate and the plain-text usually contains
      the server name $\Server$, the server's public encryption
      key and the trusted certificate authority $\Servera$. 
    \item \textbf{Third step:} the client may contact the
      server that issued the certificate and confirms that the
      certificate is valid before proceeding.
    \item \textbf{Forth step:} the certification authority sends back to the 
      client a confirmation about the credibility of the key.
    \item \textbf{Fifth step:} in order to generate the session keys used for the
      secure connection, the client encrypts a random number $N_{C'}$
      with the server's public key and sends the result to the
      server. Only the server should be able to decrypt it, using its
      private key.  
    \end{itemize}
\item Finally, from the random number
  and the session ids, both parts generate a
  session key $K_{\Client\Server}$ for encryption and decryption. This new
  knowledge will be used in the following stage.  
\end{itemize}
\input{protocols/tlsProtocol}
