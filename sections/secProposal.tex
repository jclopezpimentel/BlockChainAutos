\section{Outline}
\label{sec:outline}

%This section describes in words our general framework and describes
%the general notation used to explain the formalism.

\subsection{Outline for the Framework}
\label{subsec:proposal}
In general there exists some characteristics, in the vehicles, we are looking for:
a) Provenance,
b) Transparency with respects to purchase sale,
c) Traceability with history transactions, owners and legal situations.

To achieve that, we have proposed a distributed model illustrated in the context
diagram in Figure~\ref{fig:flowChartFramework}. Such a figure denotes
the general stages (circled numbers) a user must follow to get or set information about a 
vehicle:
\begin{figure}[bt]
 %\begin{center}
  \centering
    \includegraphics[scale=0.7]{images/gralScheme.pdf}
        \caption{Diagram about the use of the framework from the client view}
    \label{fig:flowChartFramework}
 % \end{center}
\end{figure}

\begin{itemize}
  \item \textbf{Stage I} consists in getting information of a car through reading
    a QR code by any user or setting the QR code by a dealership (this is the 
    case when a company manufactures a vehicle 
    and generates the genesis block). Details in Section~\ref{ssec:qrcode}.
  \item \textbf{Stage II} consists in establishing a secure channel between any user
    and the \blockchaincarnetwork. We
    have used the TLS protocol because it is the standard in the
    e-commerce and it has been subject to a lot of verification proofs.
    Details in Section~\ref{sec:secureChannel}.
  \item \textbf{Stage III} consists in getting services such as provenance or 
    traceability. Some services require authentication, but others do not.
    Details in Section~\ref{sec:getServices}.  
  \item \textbf{Stage IV:} consists in set transactions, for example how to build
    the genesis block within the  \blockchaincarnetwork. Authentication is required. 
    Details in Section~\ref{sec:transactions}. 
\end{itemize}

\subsection{Stage I: The QR Code of a Vehicle}
\label{ssec:qrcode}
QR code can be read quickly by many modern cell phones. It is used to take a piece of information from a 
transitory media and put it in to your cell phone. It may give you details about a URL, vCard, plain text, etc.

\subsubsection{Setting a QR Code in a Car}
\label{sssec:settingQR}
The QR code must be generated by dealerships who manufactures vehicles and they must place the code
in both: physically in the vehicle and virtually in the invoice.

We have established that the QR code must be labeled in some part of the car and it must content,  
in json format, the following data: 
\textit{id=vehicle identification number}, 
\textit{trademark}, 
\textit{model}, 
\textit{class}, 
\textit{version}, 
\textit{number of cylinders}, and
among \textit{others}. These attributes are own of a vehicle, and they do not 
change with time. An example in JSON format is as follows:
\begin{table}[h]
    \centering
    \caption{Genesis information about the vehicle}
    \begin{tabular}{lll}
       \{&         			&    							\\
         & id:        		& "1FMYU02Z97KA580G2", 			\\
         & tradeMark: 		& "abcd", 						\\
         & model:     		& "2012", 						\\
         & class:     		& "auto", 						\\
         & version:   		& "TA XLS 4X2 I4 TELA 4 CIL", 	\\
         & cylinders: 		& "L4" 							\\
       \}& 		        	& 								\\
       ::& \textit{others}	&								\\
    \end{tabular}
    \label{table:genesisInfo}
\end{table}

The information in the QR code will be in plain-text because such an information is not 
sensitive and anyone can obtain such data of any car.


\subsubsection{Reading a QR Code}
\label{sssec:readingQR}

First, we propose an application able to read a QR code (currently most smartphones contain it).
The smartphone application will be able to connect with the \blockchaincarnetwork in order to do different 
operations (See Table~\ref{table:operations} to know a list of operations). 

In particular, the smartphone application will be denoted as the client $\Client$. Such an application, 
actually, when is running connects with a miner, denoted as \Server, within the \blockchaincarnetwork.

%Throughout this document, we refer to \QR code as the information obtained after a reading 
%process. 

\input{sections/tls}

