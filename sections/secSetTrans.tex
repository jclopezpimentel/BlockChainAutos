\section{Transactions}


\subsection{Roles}

According to the permissions the users will have five defined roles: 
\textbf{U}ser,
\textbf{D}ealership,
\textbf{O}wner,
\textbf{G}overnment, and
\textbf{H}elper.
Each user could hold one or more roles.
Table \ref{table:operations} pinpoints the type of operations that each role can do and the explanation
is as follows: 
\begin{itemize}
    \item Any \textbf{U}ser will have enough permissions to interact with the system in order to 
        obtain information on the vehicle (Section~\ref{sec:getServices}). This role is also able 
        to buy cars, but it must be authenticated. 
        These permissions will be granted to the user once he has established the secure channel. 
    \item The \textbf{D}ealership is able to manufacture vehicles because of that he is who creates the 
        genesis block.
    \item Every car will have one and only one \textbf{O}wner at a given time. The owner will be the only one 
        that has permission to trade ownership, but also to request to the government to change legal
        status. In addition, he can request and authorize to \textbf{H}elper to add some information 
        about a car.
    \item \textbf{G}overnment will have the ability to  \textit{change legal status} when the 
        vehicle is involved in legal situations. For that, it will be necessary to establish 
        different status: \textit{stolen}, \textit{arrested}, \textit{penalty}, \textit{registered owner}, \textit{plate}, 
        \textit{tax history}, etc.
    \item The \textbf{H}elper role is who receives a temporal permission from the \textbf{O}wner to set one 
        transaction on the block-chain. The time frame for the temporal permission will be authorized by 
        the \textbf{O}wner of the car. The car owner will set a special transaction to give the helper 
        the permission. The helper role will be used for transactions that change one or more data of the 
        car's properties. This will include official services, mechanical adjustments, big aesthetic 
        modifications and insurance payments.
\end{itemize}







\begin{table}[htb]
\footnotesize
    \begin{center}
    \caption{Type of operation grouped by corresponding transactions}
    \label{table:operations}
        \begin{tabular}{|l|l|l|l|l|l|}
        \hline
        \textbf{Operation}          &\textbf{U}& \textbf{D}&\textbf{O}& \textbf{G}& \textbf{H}\\ \hline\hline
        Create (genesis block)      &          & X         &          &           &           \\ \hline\hline
        get information             & X        &           &          &           &           \\ \hline\hline
        sell                        &          &           & X        &           &           \\ \hline
        buy                         & X        &           &          &           &           \\ \hline\hline
        request change legal status &          &           & X        &           &           \\ \hline
        change legal status         &          &           &          & X         &           \\ \hline\hline
        request to add information  &          &           & X        &           &           \\ \hline
        add information             &          &           &          &           & X         \\ \hline
        authorize add information   &          &           & X        &           &           \\ \hline
        \end{tabular}
    \end{center}
\end{table} %added by JC


\subsection{Set Transactions} 
\label{ssec:setTrans}

As shown by Table~\ref{table:ProtSetTrans} the participants in a transaction 
are the client and a miner.
The client knows the $QR$ code through a scan process with
a car; a secret shared key $K_{\Client\Server}$ established with
a miner in the secure channel process; a proof to have been authenticated
with the miner $N_{\Client\Server}$; and the transaction $T$ that client wish 
to do  (see Table~\ref{table:operations} for types of operations). 
The miner obviously knows also $K_{\Client\Server}$ and $N_{\Client\Server}$. 
\begin{enumerate}
    \item The protocol starts when the client $\Client$ builds the transaction 
        data $t$, the signature of the transaction $\fat{t}_{K_{priv(C)}}$ and sends
        all to the miner $\Server$; all ciphered with the 
        session key $K_{\Client\Server}$.
        The transaction data $t$, using function $setTransaction(\QR,T,N_{\Client\Server})$, 
        is constituted as follows:
        
        \begin{tabular}{lll}
                &               & \\ 
            \{  &               &    \\
                & registered:   & "true" 
                & plate:        & "LVP6598", \\
                & id:           & "1FMYU02Z97KA580G2", \\
                & timestamp:    & "18/09/2018 9:16:34am", \\
                & nonce:         & $N_{\Client\Server}$, \\
                & pubKey:       & $K_{pub(X)}$, \\
                & miner:        & M, \\
                & client:       & C, \\
                & transaction:  & T=$\fat{operation}_{priv(X)}$ \\
            \}  &               &   \\
        \end{tabular}
        
        Both \textit{plate} and \textit{id} are used to identify the car, 
        \textit{timestamp} used to identify the time, $N_{\Client\Server}$
        used to distinguish the session; the public key $K_{pub(X)}$
        of the next owner client, it could be the own $\Client$ or another
        owner;
        \textit{miner} and 
        \textit{client} fields to identify the participants in the transaction; 
        and $T$ stating
        what kind of operation is carrying out (see Table~\ref{table:operations}).
    \item Having the miner accepted the previous message (verifying the  
        session with $N_{\Client\Server}$ and all received messages), he proceeds to build 
        the block $b$ and broadcasts it to all miners $\Server\rightarrow [\Server_i .. \Server_n]$
        in order to execute a mining process; 
        this message is ciphered under \hl{the Ethereum plataform...}
        Block $b$ is as follows:
        \begin{tabular}{lll}
                &               & \\ 
            \{  &               &    \\
                & dataTran:     & t,  \\
                & previousHash: & "000dc75a3...7cf", \\
                & hash:         & "000d20368...1b8",\\
                & blockId:      & 1,\\
                & nonce:        & "42543524",\\
                & timestamp:    & "18/09/2018 9:40:44am" \\
            \}  &               &   \\
        \end{tabular}
        
    \item The miner solving the challenge $\Server_x$, returns again the block, $b'$, 
        but now it is mined. 
    \item Within the \hl{Ethereum Network} there exists a parity node\footnote{
            A parity node is a miner that is close with another and they have usual
            peer to peer communication.
        } to the miner $\Server$
        who also accepts the new block $b'$.
    \item Finally, the miner responds $r$ to the client as a proof that the transaction 
        was carried out successfully.
\end{enumerate}

\begin{table}[htb]
\footnotesize
\begin{center}
\caption{Set Transactions Protocol, authentication required}
\label{table:ProtSetTrans}
\begin{tabular}{|l|}
\hline
           Initial Knowledge                                                             \\
            $C:C\lnk QR \lnk  K_{C\Server} \lnk N_{\Client\Server} \lnk T$               \\
            $\Server: \Server\lnk K_{C\Server} \lnk N_{\Client\Server}$    \\ \hline \hline 
           Set a transaction                                                                        \\
           \hspace{5mm} $t=setTransaction(QR,T,N_{\Client\Server})$                                  \\  
           1.-$\Client\rightarrow \Server: \fat{C \lnk t \lnk \fat{t}_{K_{priv(C)}}}_{K_{C\Server}}$          \\ 
           \hspace{5mm} $b=block(t)$                                  \\  
           2.-$\Server\rightarrow [\Server_i .. \Server_n]: \fat{C \lnk b \lnk N_\Server}_{K_{\Server_i .. \Server_n}}$          \\ 
           \hspace{5mm} $b'=mine(b)$                                  \\  
           3.-$\Server_x\rightarrow [\Server_i .. \Server_n]: \fat{C \lnk b'}_{K_{\Server_i .. \Server_n}}$          \\            
           \hspace{5mm} $r=f(t)$                                  \\  
           4.-$\Server\rightarrow \Client: \fat{C \lnk r \lnk N_{\Client\Server}}_{K_{C\Server}}$       \\  \hline \hline
\end{tabular}
\end{center}
\end{table}
\normalsize
