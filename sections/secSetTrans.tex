\section{Transactions}


\input{sections/roles} %added by JC


\subsection{Set Transactions} 
\label{ssec:setTrans}

As shown by Table~\ref{table:ProtSetTrans} the participants in a transaction 
are the client and a miner.
The client knows the $QR$ code through a scan process with
a car; a secret shared key $K_{\Client\Server}$ established with
a miner in the secure channel process; a proof to have been authenticated
with the miner $N_{\Client\Server}$; and the transaction $T$ that client wish 
to do  (see Table~\ref{table:operations} for types of operations). 
The miner obviously knows also $K_{\Client\Server}$ and $N_{\Client\Server}$. 
\begin{enumerate}
    \item The protocol starts when the client $\Client$ builds the transaction 
        data $t$, the signature of the transaction $\fat{t}_{K_{priv(C)}}$ and sends
        all to the miner $\Server$; all ciphered with the 
        session key $K_{\Client\Server}$.
        The transaction data $t$, using function $setTransaction(\QR,T,N_{\Client\Server})$, 
        is constituted as follows:
        
        \begin{tabular}{lll}
                &               & \\ 
            \{  &               &    \\
                & registered:   & "true" 
                & plate:        & "LVP6598", \\
                & id:           & "1FMYU02Z97KA580G2", \\
                & timestamp:    & "18/09/2018 9:16:34am", \\
                & nonce:         & $N_{\Client\Server}$, \\
                & pubKey:       & $K_{pub(X)}$, \\
                & miner:        & M, \\
                & client:       & C, \\
                & transaction:  & T=$\fat{operation}_{priv(X)}$ \\
            \}  &               &   \\
        \end{tabular}
        
        Both \textit{plate} and \textit{id} are used to identify the car, 
        \textit{timestamp} used to identify the time, $N_{\Client\Server}$
        used to distinguish the session; the public key $K_{pub(X)}$
        of the next owner client, it could be the own $\Client$ or another
        owner;
        \textit{miner} and 
        \textit{client} fields to identify the participants in the transaction; 
        and $T$ stating
        what kind of operation is carrying out (see Table~\ref{table:operations}).
    \item Having the miner accepted the previous message (verifying the  
        session with $N_{\Client\Server}$ and all received messages), he proceeds to build 
        the block $b$ and broadcasts it to all miners $\Server\rightarrow [\Server_i .. \Server_n]$
        in order to execute a mining process; 
        this message is ciphered under \hl{the Ethereum plataform...}
        Block $b$ is as follows:
        \begin{tabular}{lll}
                &               & \\ 
            \{  &               &    \\
                & dataTran:     & t,  \\
                & previousHash: & "000dc75a3...7cf", \\
                & hash:         & "000d20368...1b8",\\
                & blockId:      & 1,\\
                & nonce:        & "42543524",\\
                & timestamp:    & "18/09/2018 9:40:44am" \\
            \}  &               &   \\
        \end{tabular}
        
    \item The miner solving the challenge $\Server_x$, returns again the block, $b'$, 
        but now it is mined. 
    \item Within the \hl{Ethereum Network} there exists a parity node\footnote{
            A parity node is a miner that is close with another and they have usual
            peer to peer communication.
        } to the miner $\Server$
        who also accepts the new block $b'$.
    \item Finally, the miner responds $r$ to the client as a proof that the transaction 
        was carried out successfully.
\end{enumerate}

\input{protocols/protSetTransactions.tex}