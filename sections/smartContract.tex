\subsection{Smart Contracts}

A smart contract is a series of self-executing computer code that is fired when a transaction is trying to complete. These lines of code are part of the blockchain and allows for a series of conditions between the sender and receiver of the transaction to be completed before it succeeds. The whole process is completely automated without external help, requiring only the participation of the interested parties and the blockchain network.

The transaction itself can be a representation of a legal enforcement, and when used this way, should offer tamper-proofing to remove the possibility of bad use by either part. \cite{templates}

The existence of smart contracts over a blockchain expand the use cases of this revolutionary technology above its original use as a currency system \cite{nakamoto} \cite{uses}. Among multiple platforms that allow smart contract interactions \cite{sense}\cite{lazy}, Ethereum \cite{ethereum } as a platform also provides a full system where users can run custom-scripted code inside their transactions, embracing the idea of smart contracts over a decentralized medium, \cite{hawk}.

%To solve our model, smart properties  is a recent technology that can be used to this. 
%Smart property is those whose ownership is controlled via block chain technology using 
%smart contracts~\cite{Tapscott2016}. Examples could include physical property such as 
%vehicles, phones or houses. A smart contract is a computer protocol intended to digitally 
%facilitate, verify, or enforce the negotiation or performance of a contract, using transactions. 
%These transactions are tractable and irreversible. 

